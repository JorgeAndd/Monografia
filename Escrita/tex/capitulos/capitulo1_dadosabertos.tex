Este capítulo apresenta os conceitos e definições sobre dados abertos, em especial os governamentais e os disponibilizados pelo governo brasileiro. Na seção 2.1 é apresentada a contextualização dos dados abertos, com suas definições e características. A seção 2.2 expõe a classificação de cinco estrelas de dados abertos, descrevendo seus custos e benefícios. A seção 2.3 apresenta definições de dados abertos governamentais e seus oito princípios. Por fim, na seção 2.4 a situação dos dados abertos no contexto brasileiro é descrito.

\section{Contextualização}

O rápido crescimento no volume de dados gerados pela sociedade nos últimos anos tem levado a uma necessidade cada vez maior de ferramentas e pessoas que consigam trabalhar com esses dados. Um estudo da \emph{IDC Digital Universe}, realizado em 2011, estimou que naquele ano o volume de dados criados, replicados e consumidos foi de 1,8 trilhão de \emph{gigabytes}~\cite{gantz2012digital}. Esses dados, entretanto, não estão disponíveis de forma aberta ao público, nem estruturados de forma a facilitar a sua compreensão mesmo por aqueles que a eles tem acesso~\cite{seijiconectados}. 

Nesse contexto, diversas empresas, governos e institutos vem trabalhando para uma maior abertura desses dados, objetivando os chamados dados abertos. O termo \enquote{Dados abertos} no conceito que conhecemos hoje surgiu em 1995 em um documento de uma agência científica americana no contexto da abertura de dados geofísicos e ambientais. Esse conceito vem sendo ampliado e estimulado nos últimos anos por diversos movimentos~\cite{seijiconectados}. 

A \emph{Open Knowledge Foundation}, organização mundial que promove a abertura de dados~\cite{openknowledge}, define um dado como aberto \enquote{se qualquer pessoa está livre para acessa-lo, utiliza-lo, modifica-lo, e compartilha-lo — restrito, no máximo, a medidas que preservam a proveniência e abertura.}~\cite{opendefinition}. A abertura dos dados evita mecanismos de controle e restrições sobre os mesmos, o que permite seu uso de forma livre~\cite{seijiconectados}. Essas características podem ser resumidas nos seguintes pontos:

\begin{itemize}
\item \textbf{Disponibilidade e Acesso}: os dados devem estar disponíveis como um todo e sob custo não maior que um custo razoável de reprodução, preferencialmente possíveis de serem baixados pela internet. Os dados devem também estar disponíveis de uma forma conveniente e modificável.

\item \textbf{Reutilização e Redistribuição}: os dados devem ser fornecidos sob termos que permitam a reutilização e a redistribuição, inclusive a combinação com outros conjuntos de dados.

\item \textbf{Participação Universal}: todos devem ser capazes de usar, reutilizar e redistribuir - não deve haver discriminação contra áreas de atuação ou contra pessoas ou grupos. Por exemplo, restrições de uso 'não-comercial' que impediriam o uso 'comercial', ou restrições de uso para certos fins (ex.: somente educativos) excluem determinados dados do conceito de 'aberto'.
\end{itemize}

\section{Classificação de Dados Abertos}
Tim Berners-Lee, o inventor da \emph{Web} e um defensor da abertura de dados, propõs, em 2010, princípios que definem um sistema de classificação de dados abertos por meio de estrelas. Quanto mais aberto é o dado, maior o número de estrelas que ele possui e mais fácil é para enriquece-lo~\cite{seijiconectados}.

As cinco estrelas de Tim Berners-Lee são:
\begin{itemize}
\item \textbf{1 estrela}: O dado está disponível na Internet, em qualquer formato, acompanhado de licença aberta.
\item \textbf{2 estrelas}: O dado está disponível na Internet de maneira estruturada, em um formato que permita sua leitura por máquinas (por exemplo, XML).
\item \textbf{3 estrelas}: O dado está disponível na Internet, de maneira estruturada, e em formato não proprietário (por exemplo, \emph{CSV}).
\item \textbf{4 estrelas}: O dado está disponível na Internet, de maneira estruturada e em formato proprietário. Além disso, deve estar dentro dos padrões estabelecidos pela W3C (RDF): utilização de URI para nomear relações entre coisas e propriedades.
\item \textbf{5 estrelas}: Além das propriedades anteriores, ter no RDF conexões com outros dados, por meio de \emph{links}, de forma a se obter um contexto de dados relevantes.
\end{itemize}

Dados publicados seguindo a classificação de estrelas de Berners-Lee conferem uma série de custos e benefícios, tanto para quem os publica quanto para quem os consome, sendo eles~\cite{seijiconectados}:
\begin{itemize} 
\item \textbf{1 Estrela}: 
	\item[] \textbf{Quem consome}
		\begin{itemize}
			\itemsep0em
			\item Ver os dados
			\item Imprimi-los
			\item Guardá-los
			\item Modificar os dados como queira
			\item Acessar os dados de qualquer sistema
			\item Compartilhar os dados com qualquer pessoa			
		\end{itemize}
		
	\item[] \textbf{Quem produz}
		\begin{itemize}
			\itemsep0em
			\item É simples de publicar
			\item Não é necessário explicar repetidamente para outros que eles podem usar seus dados
		\end{itemize}
		
\item \textbf{2 Estrelas}:

	\item[] \textbf{Quem consome}
		\begin{itemize}
			\itemsep0em
			\item É possível processá-los diretamente com aplicativos proprietários 
			\item É possível exportar os dados para outro formato estruturado
		\end{itemize}
		
	\item[] \textbf{Quem produz}
		\begin{itemize}
			\itemsep0em
			\item Ainda é simples de publicar
		\end{itemize}

\item \textbf{3 Estrelas}:
	\item[] \textbf{Quem consome}
		\begin{itemize}
			\itemsep0em
			\item Pode manipular os dados da forma que quiser, sem a necessidade de nenhum \emph{software} proprietário	
		\end{itemize}
		
	\item[] \textbf{Quem produz}
		\begin{itemize}
			\itemsep0em
			\item Ainda é relativamente simples de publicar
			\item Podem ser necessários \emph{plug-ins} ou conversores para exportar os dados do \emph{software} proprietário
		\end{itemize}
\item \textbf{4 Estrelas}:
	\item[] \textbf{Quem consome}
		\begin{itemize}
			\itemsep0em
			\item Fazer marcações nos dados
			\item Reusar partes dos dados
			\item Reusar ferramentas e bibliotecas existentes
			\item Combinar os dados com outros dados
			\item O entendimento do formato RDF pode ser mais difícil do que formatos tabulares (CSV) ou em árvore (XML).
			
		\end{itemize}
		
	\item[] \textbf{Quem produz}
		\begin{itemize}
			\itemsep0em
			\item Controle granular sobre seus dados, podendo realizar otimizações de acesso
			\item Outros publicadores de dados podem fazer ligações com os dados
			\item Pode ser necessário investimento de tempo para dividir ou agrupar os dados
			
		\end{itemize}
\item \textbf{5 Estrelas}:
	\item[] \textbf{Quem consome}
		\begin{itemize}
			\itemsep0em
			\item Encontrar dados relacionados enquanto consome os dados
			\item Aprender sobre o esquema de dados	
		\end{itemize}
		
	\item[] \textbf{Quem produz}
		\begin{itemize}
			\itemsep0em
			\item Os dados podem ser encontrados com maior facilidade
			\item Os dados possuem maior valor agregado
			\item A organização ganha os mesmos benefícios da vinculação de dados que os consumidores
		\end{itemize}
\end{itemize}

\section{Dados Abertos Governamentais}
Dados governamentais, em específico, também podem ser fundamentados por três leis e oito princípios.

Em 2009 o especialista em políticas públicas e dados abertos, David Eaves, propôs as seguintes três leis dos dados abertos governamentais, que também podem ser aplicadas a dados abertos em geral~\cite{eaveslaws}:

\begin{enumerate}
\item Se o dado não pode ser encontrado e indexado na Web, ele não existe;

\item Se o dado não está disponível em um formato aberto e compreensível por máquinas, ele não pode ser reaproveitado;

\item Se algum dispositivo legal não permite que ele seja replicado, ele é inútil.

\end{enumerate}

Em 2007, um grupo de trinta especialistas em governo aberto, reunidos em Sebastopol, na California, definiu os oito princípios de dados governamentais abertos, sendo eles~\cite{opengovdata}:

\begin{itemize}
\item \textbf{Completos}: todos os dados públicos deve estar disponíveis. Dados públicos são dados que não estejam sujeitos a limitações válidas de privacidade, segurança ou privilégio.

\item \textbf{Primários}: os dados devem ser iguais aos coletados na fonte, no maior nível possível de granularidade, não estando em formas agregadas ou modificadas.

\item \textbf{Atuais}: os dados devem ser disponibilizados tão rápido quanto necessário para garantir o seu valor.

\item \textbf{Acessíveis}: os dados devem ser disponibilizados para os mais amplos públicos e propósitos possíveis.

\item \textbf{Processáveis por máquinas}: os dados devem estar estruturados de forma razoável, de forma que permita um processamento automatizado.

\item \textbf{Não discriminatórios}: os dados devem estar disponíveis para todos, sem necessidade de registro ou identificação do usuário.

\item \textbf{Não proprietários}: os dados devem estar disponíveis em um formato que não seja controlado exclusivamente por uma entidade.

\item \textbf{Livres de licença}: os dados não devem estar sujeitos a qualquer restrições de direitos autorais, marcas, patentes ou segredos industriais. Restrições razoáveis de privacidade, segurança e privilégio podem ser permitidas.

\end{itemize}

\section{Contexto Brasileiro de Dados Abertos}
A participação brasileira na área de dados abertos tem um marco em 2011 com a criação da \emph{Open Government Partnership}, uma aliança, contando inicialmente com a participação de 65 países, criada para fornecer uma plataforma internacional para reformadores nacionais comprometidos em fazer seus governos mais abertos, responsáveis e sensíveis aos cidadãos. Também foi criado o portal dados.gov.br, que disponibiliza dados governamentais de forma aberta~\cite{seijiconectados}.

O poder público brasileiro vem nos últimos anos realizando outras ações que promovem a abertura de dados governamentais. Essas ações visam benefícios como melhoria da gestão pública, transparência, controle a participação social, geração de emprego e renda e estímulo à inovação tecnológica~\cite{tcu}. Para atingir esse fim, no ano de 2012 foi definido, em instrução normativa, a implantação da INDA, Infraestrutura Nacional de Dados Abertos, \enquote{um conjunto de padrões, tecnologias, procedimentos e mecanismos de controle necessários para atender às condições de disseminação e compartilhamento de dados e informações públicas no modelo de Dados Abertos}~\cite{inda}.  

O governo tem importância fundamental na questão de dados abertos, devido à grande quantidade de dados que coleta e por serem públicos, conforme a lei, podendo ser tornados abertos e disponíveis para a sociedade~\cite{openknowledge}.

Esses dados governamentais tem relevância tanto no âmbito da transparência, podendo haver rastreamento dos impostos e gastos governamentais; da vida pessoal, como na localização de serviços públicos por parte da população; economicamente, com a reutilização de dados abertos já disponíveis; e também dentro do próprio governo, que pode aumentar sua eficiência ao permitir que a população consulte diretamente dados que antes precisavam de interferência direta e individual por parte de funcionários públicos~\cite{openknowledge}.

Apesar desses esforços e da existência de programas avançados de transparência pública, ainda são raros os órgãos que disponibilizam dados de forma aberta. Em geral esses dados estão disponíveis para visualização, mas barreiras técnicas e políticas impedem sua reutilização~\cite{w3cmanual}.








