\chapter{Introdução}

Bancos de dados podem ser definidos como um conjunto de dados que se relacionam entre si e armazenados de forma que possam ser acessados posteriormente, quando necessário.
Os bancos relacionais predominaram por pelo menos quatro décadas, mas seu desempenho em certas aplicações atuais, principalmente naquelas que trabalham com grande volumes de dados, denominado \emph{Big Data}, vem sendo questionado. 

Esse questionamento levou à criação do movimento NoSQL, um novo paradigma de armazenamento de dados que ignora certas restrições dos bancos relacionais tradicionais e tentam melhorar seu armazenamento e desempenho por meio de um sistema distribuído em \emph{clusters}, com características de escalabilidade, tolerância à falhas e um melhor desempenho ao se operar grandes volumes de dados~\cite{pramod}.

O Cassandra é atualmente um projeto da fundação \emph{Apache}, tendo sido  originalmente proposto e utilizado pelo \emph{Facebook}, com base em trabalhos anteriores da \emph{Amazon} (\emph{Dynamo}) e do \emph{Google} (\emph{BigTable}). É um banco distribuído que armazena seus dados em forma de colunas e linhas com esquema flexível. Essa característica é importante para o armazenamento de dados governamentais abertos, devido à sua natureza extremamente variável de um período à outro.

Dados abertos tem ganhado importância cada vez maior em nossa sociedade. O volume desses dados, que podem ser definidos como dados livres para acesso, utilização e modificação~\cite{opendefinition}, tem crescido cada vez mais, e vem sendo necessário encontrar novas formas para realizar o seu armazenamento e análise, comumente realizados por meio de bancos de dados.

O governo brasileiro vem, especialmente desde 2011, disponibilizando dados significativos a respeito da administração pública. Esses dados, apesar de estarem disponíveis para acesso ao público, em geral não seguem um formato que permita fácil análise. Além disso, devido ao grande volume desses dados, uma abordagem tradicional pode não ser a forma mais adequada para a sua manipulação. Esse trabalho visa analisar o desempenho de um banco de dados NoSQL no contexto dos dados abertos brasileiros, em especial os fornecidos do programa Bolsa Família pelo Portal da Transparência.

\section{Problema e Hipótese}
Vários órgãos da administração pública brasileira disponibilizam seus dados na \emph{web}. Entretanto, o grande espaço de armazenamento necessário para a totalidade desses dados pode gerar um desempenho não satisfatório ao se realizar sua inserção e posteriores consultas em um banco de dados relacional tradicional.

Temos como hipótese que a utilização de múltiplas máquinas em um ambiente Cassandra distribuído apresenta uma melhora do desempenho que justifique a sua utilização no contexto da análise de dados abertos.

\section{Justificativa}
Apesar de recente, a utilização de bancos de dados não relacionais vem apresentando um bom desempenho em aplicações que exijam a utilização de grandes volumes de dados. Porém, poucos trabalhos realizaram a utilização de bancos NoSQL no contexto dos dados abertos da administração pública. Esse trabalho possui então como motivação a análise do desempenho de um banco Cassandra ao tratar de dados abertos brasileiros.

\section{Objetivos}

\subsection{Objetivo Geral}
Esse trabalho tem como objetivo principal comparar o desempenho de bancos de dados relacionais e o de bancos NoSQL em aplicações de apoio à decisão no contexto da análise de dados abertos governamentais (INEP).

\subsection{Objetivos Específicos}
A fim de se alcançar o objetivo geral do trabalho, foram traçados os seguintes objetivos específicos:
\begin{itemize}
		\item Desenvolver uma aplicação de apoio à decisão utilizando o banco de dados NoSQL Cassandra para a análise de dados abertos governamentais(INEP).
		
		\item Comparar a performance da aplicação desenvolvida com outra já implementada utilizando a mesma arquitetura, porém com o banco de dados relacional MySQL.
		
		\item Verificar o desempenho da aplicação em um ambiente distribuído.
\end{itemize}

\section{Estrutura do Trabalho}
Esse trabalho se estrutura em:
\begin{itemize}
\item \textbf{Capítulo 2}: São apresentadas definições e características de dados abertos segundo vários autores. Os dados abertos também são contextualizados no âmbito do governo brasileiro, mostrando ações que vem sendo tomadas nesse sentido.
\item \textbf{Capítulo 3}: É feita uma exposição de conceitos relacionados a bancos de dados, realizando uma comparação entre os bancos de dados relacionais e os bancos NoSQL, além de apresentar os principais modelos de dados da abordagem NoSQL.
\item \textbf{Capítulo 4}: Detalha o banco de dados Cassandra, demonstrando suas características e funcionamento.
\end{itemize}




