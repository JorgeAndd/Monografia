

\chapter{Estudo de Caso e Metodologia}

Neste capítulo será apresentado o estudo de caso dos dados do Bolsa Família utilizado no trabalho, assim como explicadas as etapas utilizadas para a inserção e consultas desses dados em um ambiente Cassandra distribuído.

\section{Estudo de Caso}
Os dados do Bolsa Família utilizados para estudo de caso nesse trabalho foram obtidos no \emph{site} do Portal da Transparência~\cite{portaltransparencia}. Os dados são disponibilizados em arquivos mensais em formato \emph{.csv}. Para os testes realizados, foram utilizados os arquivos dos meses de ... a ... Esses arquivos possuem em média ...(tamanho) com ...(registros). Cada arquivo apresenta como campos: \textbf{UF, Código SIAFI Município, Nome Município, Código Função, Código Subfunção, Código Programa, Código Ação, NIS Favorecido, Nome Favorecido, Fonte-Finalidade, Valor Parcela e Mês Competência}. 


\section{Modelo de dados Cassandra}
O modelo de dados no Cassandra foi criado utilizando-se um \emph{keyspace} com estratégia de replicação \emph{SimpleStrategy} com fator de replicação de apenas um, pois a consistência dos dados não será um fator relevante nos testes a serem realizados.

Apenas uma família de colunas foi criada, com as colunas \textbf{UF, Código SIAFI Município, Nome Município, NIS Favorecido, Nome Favorecido, Fonte-Finalidade, Valor Parcela, Mês Competência}. Esse subconjunto de colunas foi escolhido pois os testes não buscam uma análise real das informações armazenadas, e sim mensurar o desempenho de um banco de dados não distribuido com dados públicos.

A família de colunas criada possui como chave primária os campos \textbf{NIS Favorecido, Valor Parcela e Mês Competência}, que servem para identificar unicamente cada registro, além de \textbf{UF} para permitir uma melhor distribuição dos dados no \emph{cluster}.

\section{Desenvolvimento da aplicação}
Para inserção e busca dos dados no ambiente, foi desenvolvido um programa em Java utilizando-se o driver disponibilizado pela Datastax. Essa aplicação é responsável pela leitura de todos os arquivos de entrada e a inserção dos campos utilizados no banco de dados Cassandra. A aplicação também faz a leitura desses dados inseridos a partir de uma lista de consultas a serem realizadas.

A inserção dos dados é feita de forma assíncrona utilizando os métodos do \emph{driver}. A filtragem dos campos utilizados é feita no momento da inserção pelo próprio programa desenvolvido.

A leitura dos dados é feita no mesmo programa, após a inserção de todos os dados, também utilizando o driver da Datastax.

Todas as interações com o banco, seja na inserção quanto na leitura, foram realizadas por meio da linguagem CQL, que se assemelha a consultas padrões SQL.

\section{Arquitetura do Ambiente}
Cada nós do ambiente consiste em um ...(configuração).

O ambiente foi configurado instalando-se o Cassandra em cada uma das seis máquinas utilizadas no teste. Em cada máquina foi necessário a modificação do arquivo de configuração \emph{cassandra.yaml} para possibilitar a detecção do cluster.

Além disso, seguindo as orientações em ~\cite{cassandrasettings}, guardadas as devidas limitações do laboratório, foram realizadas as seguintes configurações do Linux:
\begin{itemize}
	\item Remoção do limite de memória(\emph{memlock})
	\item Aumento do limite do número de arquivos abertos(\emph{nofile})
	\item Desativação do \emph{swap}
\end{itemize}




