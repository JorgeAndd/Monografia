

\chapter{Estudo de Caso e Metodologia}

Neste capítulo são apresentados os estudos de caso, onde os dados públicos do programa Bolsa Família são inseridos e extraídos de um banco de dados Cassandra com clusters de um, dois, quatro e seis máquinas, além das descrições das etapas e programas utilizados.

\section{Estudo de Caso}
Os dados do Bolsa Família utilizados para estudo de caso nesse trabalho foram obtidos no \emph{site} do Portal da Transparência~\cite{portaldatransparencia}. Esses dados são disponibilizados em arquivos agrupados mensalmente em formato \emph{.csv}. 

Foram utilizados no total vinte quatro arquivos, equivalentes a vinte e quatro meses do programa, compreendidos entre Julho de 2014 e Junho de 2016. Cada arquivo possui em média 16GiB de tamanho, com 13.968.749 registros.

Cada arquivo apresenta como campos: \textbf{UF, Código SIAFI Município, Nome Município, Código Função, Código Subfunção, Código Programa, Código Ação, NIS Favorecido, Nome Favorecido, Fonte-Finalidade, Valor Parcela e Mês Competência}. Um subconjunto dessas colunas foi escolhido para a realização dos testes, como explicado na próxima sessão.

\section{Modelo de dados Cassandra}
O modelo de dados no Cassandra foi criado utilizando-se um \emph{keyspace} com estratégia de replicação \emph{SimpleStrategy} com fator de replicação de apenas um, pois a tolerância a falhas não é um fator relevante nos testes a serem realizados, devido ao ambiente controlado em que eles são realizados.

Apenas uma família de colunas foi criada, com as colunas \textbf{UF, Código SIAFI Município, Nome Município, NIS Favorecido, Nome Favorecido, Fonte-Finalidade, Valor Parcela, Mês Competência}. Esse subconjunto de colunas foi escolhido pois os testes não buscam uma análise real das informações armazenadas, e sim mensurar o desempenho de um banco de dados distribuido com dados públicos.

A família de colunas criada possui como chave primária os campos \textbf{NIS Favorecido, Valor Parcela e Mês Competência}, que servem para identificar unicamente cada registro, já que um mesmo favorecido pelo programa(NIS Favorecido) pode receber benefícios em meses diferentes, e também mais de uma parcela em um mesmo mês. Além disso, para garantir uma melhor distribuição dos dados no \emph{cluster}, foi adicionada a coluna \textbf{UF}. 

% TODO: Distribuição de dados no cluster. Por que essa escolha de chave?

A configuração do ambiente Cassandra foi feita por meio do cliente \emph{cqlsh}, que permite realizar consultas e manipulações no banco por meio da linguagem \emph{CQL}(\emph{Cassandra Query Language}).

\section{Desenvolvimento da aplicação}
Para inserção e busca dos dados no ambiente, foi desenvolvida uma aplicação em Java utilizando o \emph{driver} disponibilizado pela \emph{Datastax}. Essa aplicação é responsável pela leitura de todos os arquivos de entrada e a inserção dos campos utilizados no banco de dados Cassandra, assim como a posterior busca dos dados.

A inserção dos dados é feita de forma assíncrona utilizando os métodos do \emph{driver}. A filtragem dos campos utilizados é feita no momento da inserção pelo próprio programa desenvolvido.

A leitura dos dados é feita no mesmo programa, após a inserção de todos os dados, também utilizando o driver da \emph{Datastax}.

Todas as interações com o banco, seja na inserção quanto na leitura, foram realizadas por meio da linguagem CQL, que se assemelha a consultas padrões SQL.

\section{Arquitetura do Ambiente}
O ambiente utilizado consiste em seis máquinas com Intel i5-4570 3.20GHz, 16GB de RAM, disco rígido de 500GB, com sistema operacional Ubuntu.

O cliente Cassandra foi instalado em cada uma das seis máquinas utilizadas no teste. Em cada máquina foi necessário a modificação do arquivo de configuração \emph{cassandra.yaml} para possibilitar a detecção do \emph{cluster}.

Além disso, seguindo as orientações em ~\cite{cassandrasettings}, e guardadas as devidas limitações do laboratório, foram realizadas as seguintes configurações do Linux:
\begin{itemize}
	\item Remoção do limite de memória(\emph{memlock})
	\item Aumento do limite do número de arquivos abertos(\emph{nofile})
	\item Desativação do \emph{swap}
\end{itemize}




