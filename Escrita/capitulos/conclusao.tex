\chapter{Conclusão e Trabalhos Futuros}

Esta monografia apresentou um estudo de caso para análise do desempenho de um banco de dados não relacional, especificamente o Cassandra, para armazenamento e análise dados públicos abertos, sendo para isso utilizados os dados dos pagamentos do programa Bolsa Família. A análise foi feita com a criação de um modelo Cassandra, criação de modelo de dados e testes de inserção e busca com diferentes configurações de \emph{cluster} e volume de dados.

Foram testados quatro configurações diferentes de ambientes, com duas, quatro e seis máquinas, e dois volumes de dados. A inserção e a busca de dados foi realizada por meio de uma aplicação Java desenvolvida com utilização dos \emph{drivers} disponibilizados pela \emph{Datastax}. Foi feita uma comparação entre os diferentes resultados de performance, tanto na inserção quanto na busca de dados, obtendo-se uma melhora média de x\% na inserção e de x\% na busca de dados.

Apesar dos testes com Cassandra apresentarem uma melhora de desempenho com o aumento do número de máquinas, possíveis limitações de rede no ambiente utilizado não permitiram uma análise isolada da performance com aumento do número de máquina. Um possível trabalho futuro poderia tentar isolar as máquinas utilizadas em uma rede dedicada, de forma a analisar o desempenho em um ambiente de servidor mais realista. Também seria interessante realizar a mesma análise com outros bancos de dados, especialmente com o uso de um banco de dados relacional convencional, de forma a se verificar uma real vantagem de um banco de dados Cassandra para o volume de dados utilizado.

Além disso, a modelagem de um banco Cassandra, por ser baseado nas consultas dos dados, possui forte influência sobre o desempenho da aplicação. Outros trabalhos futuros poderiam explorar diferentes modelagens para as tabelas, incluindo consultas diferenciadas por modelagem.