\chapter{Dados Abertos}

A \emph{Open Definition} define um dado como aberto \enquote{se qualquer pessoa esta livre para acessa-lo, utiliza-lo, modifica-lo, e compartilha-lo — restrito, no maximo, a medidas que preservam a proveniência e abertura.}~\cite{opendefinition}. A abertura dos dados evita mecanismos de controle e restrições sobre esses dados, o que permite seu uso de forma livre~\cite{seijiconectados}.

A Open Knowledge Foundation, organização mundial que promove a abertura de dados~\cite{openknowledge}, resume esses pontos como:

\begin{itemize}
\item \textbf{Disponibilidade e Acesso}: os dados devem estar disponíveis como um todo e sob custo não maior que um custo razoável de reprodução, preferencialmente possíveis de serem baixados pela internet. Os dados devem também estar disponíveis de uma forma conveniente e modificável.

\item \textbf{Reutilização e Redistribuição}: os dados devem ser fornecidos sob termos que permitam a reutilização e a redistribuição, inclusive a combinação com outros conjuntos de dados.

\item \textbf{Participação Universal}: todos devem ser capazes de usar, reutilizar e redistribuir - não deve haver discriminação contra áreas de atuação ou contra pessoas ou grupos. Por exemplo, restrições de uso 'não-comercial' que impediriam o uso 'comercial', ou restrições de uso para certos fins (ex.: somente educativos) excluem determinados dados do conceito de 'aberto'.
\end{itemize}

O termo \enquote{Dados abertos} surgiu em 1995 em um documento de uma agência científica americana, e esse conceito vem sendo ampliado e estimulado nos últimos anos por diversos movimentos~\cite{seijiconectados}. 

A participação brasileira tem um marco em 2011 com a criação da \emph{Open Government Partnership}, contando inicialmente com a participação de 65 países. Também foi criado o portal dados.gov.br, que disponibiliza dados governamentais de forma aberta~\cite{seijiconectados}.


O poder público brasileiro vem nos últimos anos realizando outras ações que promovem a abertura de dados governamenais. Essas ações visam benefícios como melhoria da gestão pública, transparência, controle a participação social, geração de emprego e renda e estímulo à inovação tecnológica~\cite{tcu}. Para atingir esse fim, no ano de 2012 foi definido, em instrução normativa, a implantação da INDA, Infraestrutura Nacional de Dados Abertos, \enquote{um conjunto de padrões, tecnologias, procedimentos e mecanismos de controle necessarios para atender às condições de disseminação e compartilhamento de dados e informações públicas no modelo de Dados Abertos}~\cite{inda}.  

O governo tem importancia fundamental na questão de dados abertos, devido â grande quantidade de dados que coleta e por serem públicos, conforme a lei, podendo ser tornados abertos e disponíveis para a sociedade~\cite{openknowledge}.

Esses dados governamentais tem relevância tanto no âmbito da transparência, podendo haver rastreamento dos impostos e gastos governamentais; da vida pessoal, como na localização de serviços públicos por parte da população; economicamente, com a reutilização de dados abertos ja disponíveis; e também dentro do próprio governo, que pode aumentar sua eficiência ao permitir que a população consulte diretamente dados que antes precisavam de interferência direta e individual por parte de funcionarios públicos~\cite{openknowledge}.









