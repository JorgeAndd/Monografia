\usepackage[american]{babel}
\usepackage{amsmath}
\usepackage{amssymb}
\usepackage{amsthm}

%%%%%%%%%%%%%%%%%%%%%%%%%%%%%%%%%%%%%%%%%%%%%%%%%%%%%%%%%%%%%%%%%%%%%%
% Codificação
%%%%%%%%%%%%%%%%%%%%%%%%%%%%%%%%%%%%%%%%%%%%%%%%%%%%%%%%%%%%%%%%%%%%%%

% Para LaTeX
\usepackage[T1]{fontenc}
\usepackage[utf8]{inputenc}
\usepackage[final,expansion=true,protrusion=true,spacing=true,kerning=true]{microtype}

% Para XeLaTeX
%\usepackage{fontspec}
%\usepackage{xunicode}
%\usepackage{xltxtra}


%%%%%%%%%%%%%%%%%%%%%%%%%%%%%%%%%%%%%%%%%%%%%%%%%%%%%%%%%%%%%%%%%%%%%%
% Pacotes usados
%%%%%%%%%%%%%%%%%%%%%%%%%%%%%%%%%%%%%%%%%%%%%%%%%%%%%%%%%%%%%%%%%%%%%%

% Importar pacotes
\usepackage{setspace}
\usepackage{xcolor}
\usepackage{array}
%\usepackage{longtable}
%\usepackage{cancel}
\usepackage{graphicx}
%\usepackage[colorlinks,pdfauthor={Felipe Lessa}]{hyperref}
\usepackage{url}
\usepackage{enumerate}
%\usepackage{tikz}
%\usepackage{subfig}
%\usepackage{pstricks}
%\usepackage{pst-pdf}
%\usepackage{vaucanson-g}
%\usepackage{qtree}
%\usepackage{listings}
%\usepackage{rotating}
%\usepackage[alf,abnt-emphasize=bf,abnt-url-package=hyperref]{abntcite}
\bibliographystyle{abnt-num}

% Caminho para as imagens
\graphicspath{{../images/}}


%%%%%%%%%%%%%%%%%%%%%%%%%%%%%%%%%%%%%%%%%%%%%%%%%%%%%%%%%%%%%%%%%%%%%%
% Fontes
%%%%%%%%%%%%%%%%%%%%%%%%%%%%%%%%%%%%%%%%%%%%%%%%%%%%%%%%%%%%%%%%%%%%%%

% Para LaTeX
%\usepackage{times,mathptmx}
\usepackage{charter,mathpazo}
%\usepackage{pxfonts}
%\usepackage{courier}
%\usepackage{rotating}

% Para XeLaTeX
%\defaultfontfeatures{Mapping=tex-text}
%\setmainfont[Scale=0.95, Ligatures={Common}, Numbers={Lining}]{URW Palladio L}%{Cantarell}
%\setsansfont[Scale=0.95, Ligatures={Common}, Numbers={Lining}]{URW Palladio L}%{Cantarell}
%\setmonofont[Scale=0.9]{Liberation Mono}
%\usefonttheme[onlymath]{serif}

\renewcommand\emph{\textbf}

%%%%%%%%%%%%%%%%%%%%%%%%%%%%%%%%%%%%%%%%%%%%%%%%%%%%%%%%%%%%%%%%%%%%%%
% Cores
%%%%%%%%%%%%%%%%%%%%%%%%%%%%%%%%%%%%%%%%%%%%%%%%%%%%%%%%%%%%%%%%%%%%%%

% Tango
\definecolor{LightButter}{rgb}{0.98,0.91,0.31}
\definecolor{LightOrange}{rgb}{0.98,0.68,0.24}
\definecolor{LightChocolate}{rgb}{0.91,0.72,0.43}
\definecolor{LightChameleon}{rgb}{0.54,0.88,0.20}
\definecolor{LightSkyBlue}{rgb}{0.45,0.62,0.81}
\definecolor{LightPlum}{rgb}{0.68,0.50,0.66}
\definecolor{LightScarletRed}{rgb}{0.93,0.16,0.16}
\definecolor{Butter}{rgb}{0.93,0.86,0.25}
\definecolor{Orange}{rgb}{0.96,0.47,0.00}
\definecolor{Chocolate}{rgb}{0.75,0.49,0.07}
\definecolor{Chameleon}{rgb}{0.45,0.82,0.09}
\definecolor{SkyBlue}{rgb}{0.20,0.39,0.64}
\definecolor{Plum}{rgb}{0.46,0.31,0.48}
\definecolor{ScarletRed}{rgb}{0.80,0.00,0.00}
\definecolor{DarkButter}{rgb}{0.77,0.62,0.00}
\definecolor{DarkOrange}{rgb}{0.80,0.36,0.00}
\definecolor{DarkChocolate}{rgb}{0.56,0.35,0.01}
\definecolor{DarkChameleon}{rgb}{0.30,0.60,0.02}
\definecolor{DarkSkyBlue}{rgb}{0.12,0.29,0.53}
\definecolor{DarkPlum}{rgb}{0.36,0.21,0.40}
\definecolor{DarkScarletRed}{rgb}{0.64,0.00,0.00}
\definecolor{Aluminium1}{rgb}{0.93,0.93,0.92}
\definecolor{Aluminium2}{rgb}{0.82,0.84,0.81}
\definecolor{Aluminium3}{rgb}{0.73,0.74,0.71}
\definecolor{Aluminium4}{rgb}{0.53,0.54,0.52}
\definecolor{Aluminium5}{rgb}{0.33,0.34,0.32}
\definecolor{Aluminium6}{rgb}{0.18,0.20,0.21}

% Cor dos links
\hypersetup{
	linkcolor=DarkScarletRed,
	citecolor=DarkScarletRed,
	filecolor=DarkScarletRed,
	urlcolor= DarkScarletRed
}


%%%%%%%%%%%%%%%%%%%%%%%%%%%%%%%%%%%%%%%%%%%%%%%%%%%%%%%%%%%%%%%%%%%%%%
% Tema do beamer
%%%%%%%%%%%%%%%%%%%%%%%%%%%%%%%%%%%%%%%%%%%%%%%%%%%%%%%%%%%%%%%%%%%%%%

%\usetheme{Madrid}
%\usetheme{Singapore}
%\usetheme[height=2em]{Rochester}
\usetheme{Warsaw}


%%%%%%%%%%%%%%%%%%%%%%%%%%%%%%%%%%%%%%%%%%%%%%%%%%%%%%%%%%%%%%%%%%%%%%
% Informações da apresentação
%%%%%%%%%%%%%%%%%%%%%%%%%%%%%%%%%%%%%%%%%%%%%%%%%%%%%%%%%%%%%%%%%%%%%%

\title[Sistema de Apoio à Decisão com uso de NoSQL]{Sistema de Apoio à Decisão de Dados Abertos com uso de Bancos de Dados NoSQL}
\author{Jorge Luiz Andrade}
\institute[UnB]{Universidade de Brasília}
\date{13 de junho de 2016}


% Mostrar sumário entre seções
\AtBeginSubsection[]
{
  \begin{frame}<beamer>{Summary}
    \tableofcontents[currentsection,currentsubsection]
  \end{frame}
}
