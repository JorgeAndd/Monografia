%%%%%%%%%%%%%%%%%%%%%%%%%%%%%%%%%%%%%%%%%%%%%%%%%%%%%%%%%%%%%%%%%%%%%%
% Slides
%%%%%%%%%%%%%%%%%%%%%%%%%%%%%%%%%%%%%%%%%%%%%%%%%%%%%%%%%%%%%%%%%%%%%%

\begin{frame}
\titlepage
\end{frame}

\begin{frame}{Roteiro}
  \tableofcontents
\end{frame}

\section{Introdução}

\begin{frame}{Cassandra}
\begin{itemize}

\item Distribuido e Descentralizado
\item Elasticamente Escalável
\item Altamente Disponível e Tolerante a Falhas
\item Variavelmente Consistente

\end{itemize}
\end{frame}

\section{Problema e Hipótese}

\begin{frame}{Contexto}
    \begin{itemize}
      \item Vários órgãos da administração pública brasileira disponibilizam seus dados na web;
      
      \item O formato original e a falta de integração dificultam a análise desses dados.
  \end{itemize}
\end{frame}

\begin{frame}{Problema}
  \begin{block}{Problema}
Bancos de dados não relacionais (NoSQL) podem melhorar o desempenho de sistemas de apoio à decisão baseados em uma arquitetura OLAP para a análise de uma grande massa de dados abertos governamentais?
  \end{block}
\end{frame}

\begin{frame}{Hipótese}

\begin{block}{Hipótese}
A utilização de sistemas gerenciadores de bancos de dados NoSQL em sistemas de apoio à decisão baseados em arquitetura OLAP para a análise de grandes massas de dados pode apresentar uma performance melhor que a utlização de sistemas gerenciadores de banco de dados relacionais.
\end{block}

\end{frame}

\section{Objetivos}

\begin{frame}{Objetivos}

    \begin{block}{Geral}
    Comparar o desempenho de bancos de dados relacionais e NoSQL em aplicações de apio à decisão baseadas em uma arquitetura OLAP para a análise de dados abertos governamentais.
    \end{block}
    
\end{frame}

\begin{frame}{Objetivos}

    \begin{block}{Específicos}
    \begin{itemize}
        \item Desenvolver uma aplicação de apoio à decisão baseada em uma arquitetura OLAP utilizando o banco de dados NoSQL Cassandra para a análise de dados abertos governamentais.
        
        \item Comparar a performance da aplicação desenvolvida com outra já implementada por alunos de semestres anteriores utilizando a mesma arquitetura, porém com o banco de dados relacional MySQL. Cabe ressaltar que as duas aplicações utilizarão o mesmo conjunto de dados.
    \end{itemize}
    \end{block}

\end{frame}

\begin{frame}{Resultados Esperados}

Espera-se obter uma aplicação com melhor desempenho em relação à uma arquitetura convencional OLAP, baseada em bancos de dados relacionais, em um ambiente passível de escalabilidade.
  
\end{frame}

\section{Metodologia}

\begin{frame}{Metodologia}

\begin{itemize}

\item Estudo bibliográfico
\item Estudo do modelo de dados
\item Implementação da ferramenta 
\item Execução e testes
\item Análise e comparação dos resultados

\end{itemize}

\end{frame}


\begin{frame}{Cronograma}

\begin{table}[]
\centering
\resizebox{\textwidth}{!}{%
\begin{tabular}{|>{\bfseries}l|c|c|c|c|c|c|c|}
\hline
\textbf{Atividade} & \multicolumn{1}{l|}{\textbf{Junho}} & \multicolumn{1}{l|}{\textbf{Julho}} & \multicolumn{1}{l|}{\textbf{Agosto}} & \multicolumn{1}{l|}{\textbf{Setembro}} & \multicolumn{1}{l|}{\textbf{Outubro}} & \multicolumn{1}{l|}{\textbf{Novembro}} & \multicolumn{1}{l|}{\textbf{Dezembro}} \\ \hline
\textbf{Estudo bibliográfico} & x & x & x &  &  &  &  \\ \hline
\textbf{Estudo do modelo de dados} &  & x & x &  &  &  &  \\ \hline
\textbf{Implementação da ferramenta} &  &  & x & x & x &  &  \\ \hline
\textbf{Execução e testes} &  &  &  &  & x & x &  \\ \hline
\textbf{\begin{tabular}[c]{@{}l@{}}Análise e comparação \\ dos resultados\end{tabular}} &  &  &  &  &  & x & x \\ \hline
\end{tabular}%
}
\end{table}
\end{frame}

\section{Conclusão}

\appendix

\begin{frame}
  \frametitle{Obrigado pela atenção!}
  \begin{center}
    {\Huge Obrigado!}
  \end{center}
\end{frame}

\begin{frame}[allowframebreaks]
  \frametitle{Créditos}
  \begin{description}
  \item[Slide~\pageref{fig:exemplo}]
    \url{http://hackage.haskell.org/package/hierarchical-clustering-diagrams}.
  \end{description}
\end{frame}
