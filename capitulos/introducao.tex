\chapter{Introdução}

Dados abertos tem ganhado importância cada vez maior em nossa sociedade. O volume desses dados, que podem ser definidos como dados livres para acesso, utilização e modificação~\cite{opendefinition}, tem crescido cada vez mais, e vem sendo necessário encontrar novas formas para realizar o seu armazenamento e analise, comumente realizados por meio de bancos de dados.

Bancos de dados podem ser definidos como um conjunto de dados que se relazionam entre si e armazenados de forma que possam ser acessados posteriormente, quando necessario~\cite{leavitt2010nosql}.
Os bancos de dados relacionais predominaram por pelo menos nas últimas três décadas, mas seu desempenho em certas aplicações atuais, principalmente naquelas que trabalham com grande volumes de dados, denominado \emph{Big Data}, vem sendo questionado. 

Esse questionamento levou à criação do movimento NoSQL, um novo paradigma de armazenamento de dados que ignora certas restrições dos bancos relacionais tradicionais e tentam melhorar seu armazenamento e desempenho por meio de um sistema distribuído em \emph{clusters}, com características de escalabilidade, tolerância à falhas e um melhor desempenho ao se operar grandes volumes de dados.

A utilização de bancos NoSQL para o tratamento de grande volumes de dados provenientes de dados abertos é uma possibilidade a ser analisada, a fim de permitir um acesso mais fácil e rápido por parte da população. 

Vários modelos NoSQL são descritos nesse trabalho, e o banco Cassandra, um representante de bancos orientados à colunas, foi utilizado. O Cassandra é atualmente um projeto da fundação \emph{Apache}, tendo sido  originalmente proposto e utilizado pelo \emph{Facebook}, com base em trabalhos anteriores da \emph{Amazon} (\emph{Dynamo}) e do \emph{Google} (\emph{BigTable}). É um banco distribuído que armazena seus dados em forma de colunas e linhas com esquema flexível. Essa características é importante para o armazenamento de dados governamentais abertos, devido à sua natureza extremamente variável de um período à outro.

\section{Problema e Hipótese}
Vários órgãos da administração pública brasileira disponibilizam seus dados na web. No entanto, o formato original e a falta de integração dificultam a análise desses dados. Nesse contexto, esse trabalho se propõe a analisar se bancos de dados não relacionais (NoSQL) podem melhorar o desempenho de sistemas de apoio à decisão baseados em uma arquitetura Online Analytical Processing (OLAP) para a análise de uma grande massa de dados abertos governamentais.

Temos como hipótese que a utilização de bancos NoSQL podem apresentar um desempenho superior ao alcançado com a utilização de sistemas gerenciadores de bancos de dados relacionais.

\section{Justificativa}
Um sistema que facilite a análise de dados abertos, em especial em grandes volumes e governamentais, pode trazer diversos benefícios à sociedade.

\section{Objetivos}

\subsection{Objetivo Geral}
Esse trabalho tem como objetivo principal comparar o desempenho de bancos de dados relacionais e o de bancos NoSQL em aplicações de apoio à decisão no contexto da análise de dados abertos governamentais(INEP).

\subsection{Objetivos Específicos}
\begin{itemize}
		\item Desenvolver uma aplicação de apoio à decisão utilizando o banco de dados NoSQL Cassandra para a análise de dados abertos governamentais(INEP).
		
		\item Comparar a performance da aplicação desenvolvida com outra já implementada utilizando a mesma arquitetura, porém com o banco de dados relacional MySQL.
		
		\item Verificar o desempenho da aplicação em um ambiente distribuído.
\end{itemize}





