\chapter{Introdução}

Dados abertos tem ganhado importância cada vez maior em nossa sociedade. O volume desses dados, que podem ser definidos como dados livres para acesso, utilização e modificação~\cite{opendefinition}, tem crescido cada vez mais, e vem sendo necessário encontrar novas formas para realizar o seu armazenamento e análise, comumente realizados por meio de bancos de dados.

Bancos de dados podem ser definidos como um conjunto de dados que se relazionam entre si e armazenados de forma que possam ser acessados posteriormente, quando necessário~\cite{leavitt2010nosql}.
Os bancos de dados relacionais predominaram por pelo menos nas últimas três décadas, mas seu desempenho em certas aplicações atuais, principalmente naquelas que trabalham com grande volumes de dados, denominado \emph{Big Data}, vem sendo questionado. 

Esse questionamento levou à criação do movimento NoSQL, um novo paradigma de armazenamento de dados que ignora certas restrições dos bancos relacionais tradicionais e tentam melhorar seu armazenamento e desempenho por meio de um sistema distribuído em \emph{clusters}.

A utilização de bancos NoSQL para o tratamento de grande volumes de dados provenientes de dados abertos é uma possibilidade a ser analisada, a fim de permitir um acesso mais fácil e rápido por parte da população. 








