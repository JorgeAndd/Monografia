\chapter{Dados Abertos}

O termo \enquote{Dados abertos} surgiu em 199,5 em um documento de uma agência científica americana, e seu conceito vem sendo ampliado e estimulado nos últimos anos por diversos movimentos~\cite{seijiconectados}. 

A \emph{Open Definition} define um dado como aberto \enquote{se qualquer pessoa está livre para acessa-lo, utiliza-lo, modifica-lo, e compartilha-lo — restrito, no maximo, a medidas que preservam a proveniência e abertura.}~\cite{opendefinition}. A abertura dos dados evita mecanismos de controle e restrições sobre os mesmos, o que permite seu uso de forma livre~\cite{seijiconectados}.

A Open Knowledge Foundation, organização mundial que promove a abertura de dados~\cite{openknowledge}, resume esses pontos em:

\begin{itemize}
\item \textbf{Disponibilidade e Acesso}: os dados devem estar disponíveis como um todo e sob custo não maior que um custo razoável de reprodução, preferencialmente possíveis de serem baixados pela internet. Os dados devem também estar disponíveis de uma forma conveniente e modificável.

\item \textbf{Reutilização e Redistribuição}: os dados devem ser fornecidos sob termos que permitam a reutilização e a redistribuição, inclusive a combinação com outros conjuntos de dados.

\item \textbf{Participação Universal}: todos devem ser capazes de usar, reutilizar e redistribuir - não deve haver discriminação contra áreas de atuação ou contra pessoas ou grupos. Por exemplo, restrições de uso 'não-comercial' que impediriam o uso 'comercial', ou restrições de uso para certos fins (ex.: somente educativos) excluem determinados dados do conceito de 'aberto'.
\end{itemize}

\section{Dados abertos governamentais}
Dados governamentais, em específico, também podem ser fundamentados por três leis e oito princípios.

Em 2009 o especialista em políticas públicas e dados abertos, David Eaves, propôs as seguintes três leis dos dados abertos governamentais, que também podem ser aplicadas a dados abertos em geral~\cite{eaveslaws}:

\begin{enumerate}
\item Se o dado não pode ser encontrado e indexado na Web, ele não existe;

\item Se o dado não está disponível em um formato aberto e compreensível por máquinas, ele não pode ser reaproveitado;

\item Se algum dispositivo legal não permite que ele seja replicado, ele é inútil.

\end{enumerate}

Em 2007, um grupo de trinta especialistas em governo aberto, reunidos em Sebastopol, na California, definiu os oito princípios de dados governamentais abertos, sendo eles~\cite{opengovdata}:

\begin{itemize}
\item \textbf{Completos}: todos os dados públicos deve estar disponíveis. Dados públicos são dados que não estejam sujeitos a limitações válidas de privacidade, segurança ou privilégio.

\item \textbf{Primários}: os dados devem ser iguais aos coletados na fonte, no maior nível possível de granularidade, não estando em formas agregadas ou modificadas.

\item \textbf{Atuais}: os dados devem ser disponibilizados tão rápido quanto necessário para garantir o seu valor.

\item \textbf{Acessíveis}: os dados devem ser disponibilizados para os mais amplos públicos e propósitos possíveis.

\item \textbf{Processáveis por máquinas}: os dados devem estar estruturados de forma razoável, de forma que permita um processamento automatizado.

\item \textbf{Não discriminatórios}: os dados devem estar disponíveis para todos, sem necessidade de registro ou identificação do usuário.

\item \textbf{Não proprietários}: os dados devem estar disponíveis em um formato que não seja controlado exclusivamente por uma entidade.

\item \textbf{Livres de licença}: os dados não devem estar sujeitos a qualquer restrições de direitos autorais, marcas, patentes ou segredos industriais. Restrições razoáveis de privacidade, segurança e privilégio podem ser permitidas.

\end{itemize}

\section{Contexto brasileiro de dados abertos}
A participação brasileira na área de dados abertos tem um marco em 2011 com a criação da \emph{Open Government Partnership}, uma aliança, contando inicialmente com a participação de 65 países, criada para fornecer uma plataforma internacional para reformadores nacionais comprometidos em fazer seus governos mais abertos, responsáveis e sensíveis aos cidadãos. Também foi criado o portal dados.gov.br, que disponibiliza dados governamentais de forma aberta~\cite{seijiconectados}.

O poder público brasileiro vem nos últimos anos realizando outras ações que promovem a abertura de dados governamenais. Essas ações visam benefícios como melhoria da gestão pública, transparência, controle a participação social, geração de emprego e renda e estímulo à inovação tecnológica~\cite{tcu}. Para atingir esse fim, no ano de 2012 foi definido, em instrução normativa, a implantação da INDA, Infraestrutura Nacional de Dados Abertos, \enquote{um conjunto de padrões, tecnologias, procedimentos e mecanismos de controle necessarios para atender às condições de disseminação e compartilhamento de dados e informações públicas no modelo de Dados Abertos}~\cite{inda}.  

O governo tem importancia fundamental na questão de dados abertos, devido â grande quantidade de dados que coleta e por serem públicos, conforme a lei, podendo ser tornados abertos e disponíveis para a sociedade~\cite{openknowledge}.

Esses dados governamentais tem relevância tanto no âmbito da transparência, podendo haver rastreamento dos impostos e gastos governamentais; da vida pessoal, como na localização de serviços públicos por parte da população; economicamente, com a reutilização de dados abertos ja disponíveis; e também dentro do próprio governo, que pode aumentar sua eficiência ao permitir que a população consulte diretamente dados que antes precisavam de interferência direta e individual por parte de funcionarios públicos~\cite{openknowledge}.









