\chapter{Cassandra}

Esse trabalho ira utilizar o Cassandra como banco de dados para validação de sua hipótese. Como visto no capítulo 3, o Apache Cassandra, distribuição que sera utilizada, é um banco de dados orientado a colunas altamente disponível e distribuído em servidores constituídos de hardware de \enquote{prateleira} para gerenciamento de grande volumes de dados~\cite{lakshmancassandra}. Este capítulo tem como objetivo definir esse banco de dados, suas características, funcionamento, vantagens e desvantagens.

\section{Definição}
O Cassandra se originou em 2007 como um projeto do \emph{Facebook} para resolver um problema na busca da caixa de mensagens. A compania necessitava de um sistema com alta performance, confiabilidade, eficiência e que suportasse o contínuo crescimento da ferramenta~\cite{lakshmancassandra, cassandraguide}. 

O projeto foi desenvolvido por Jeff Hammerbacher, Avinash Lakshman, Karthik Ranganathan e Prashant Malik, tendo seu modelo de dados sofrido grande inspiração nos trabalhos anteriores do \emph{Amazon Dynamo}~\cite{dynamo} e do \emph{Google Bigtable}~\cite{bigtable}, e lançado em 2008 como um projeto \emph{open source}. Foi mantido e atualizado apenas pelo Facebook até 2009, quando foi comprado pela Apache~\cite{cassandraguide}, sendo utilizado atualmente por companias como \emph{Netflix}, \emph{Spotify} e até em agências governamentais, como a NASA~\cite{cassandracompanies}. 

Pode ser definido como um banco de dados orientado a colunas \emph{open source}, distribuído, descentralizado, elasticamente escalavel, altamente disponível, tolerante a falhas e variavelmente consistente~\cite{cassandraguide}. A seguir iremos analisar cada uma dessas características.

\subsection{Características}

\subsection*{Distribuído e Descentralizado}
O Cassandra é capaz de ser executado em múltiplas de forma transparente ao usuário, que o enxerga como um sistema unificado. Apesar de ser possível sua execução em um único nó, só é possível obter algum benefício com uma execução distribuída. Além do ganho de performance, a distributividade do sistema garante maior segurança devido à redundância de dados.

Diferente de outros bancos distribuidos que elegem nós como mestres e escravos, o Cassandra opera de forma descentralizada, o que significa que todos os nós são idênticos em sua forma de execução, sendo utilizados protocolos \emph{peer-to-peer} (par-a-par) e \emph{gossip} para manutenção e sincronia entre os nós. Essa descentralização garante que não exista apenas um ponto de falha, o que aumenta sua disponibilidade, e simplifica a operação do e manutenção do \emph{cluster}.

\subsection*{Elasticamente Escalável}
Escalabilidade é a propriedade que um sistema tem de atender um crescente número de requisições sem prejuízo de performance. Essa escalabilidade pode ser tanto vertical quanto horizontal. Na escalabilidade vertical o hardware já utilizado no sistema é melhorado, enquanto na escalabilidade horizontal novos máquinas são adicionadas à arquitetura, havendo a divisão da carga do sistema.

O Cassandra possui escalabilidade horizontal elástica, o que significa que sua arquitetura pode escalar tanto para cima quanto para baixo. Na necessidade de uma melhora do desempenho da aplicação, novas máquinas podem ser adicionadas, e o Cassandra se encarrega de fazer a distribuição dos dados de forma transparente, sem necessidade de configurações adicionais ou reiniciamento do sistema. Da mesma forma, em caso de necessidade, máquinas podem ser retiradas do \emph{cluster} sem prejuizo ao todo, devido ao rebalanceamento automático.

\subsection*{Altamente disponível e Tolerante a falhas}
A disponibilidade de um sistema é medida de acordo com sua capacidade de responder a requisições. Computadores, e especialmente sistemas distribuídos em rede, estão sujeitos a falhas, que em geral só podem ser contornadas por meio de sistemas redundantes.

Devido a replicação e redundância de dados e a sua capacidade de substituição de nós indisponíveis, o Cassandra pode ser definido como um sistema altamente disponível e tolerante à falhas em suas máquinas.

\subsection*{Variavelmente Consistente}
A consistência de uma aplicação diz respeito à sua capacidade de retornar o valor mais atual em uma requisição.

Como visto no Teorema CAP\ref{sec:cap}, não é possível a um sistema ser totalmente conscistente, disponível e tolerante a falhas. 

O Cassandra é por vezes definido como \enquote{eventualmente consistente}, por trocar parte de sua consistência por alta disponibilidade. Essa definição, porém, não é totalmente correta, e um termo melhor para defini-lo é \enquote{variavelmente consistente} (\emph{tuneably consistent}), podendo essa sua consistência ser ajustada de acordo com o tipo de aplicação.

\section{Modelo de Dados}

Um banco de dados Cassandra consiste em um \emph{keyspace}, formado por colunas agrupadas em conjuntos chamados famílias de colunas. Esse modelo é bastante semelhante ao que foi proposto pelo Bigtable~\cite{lakshmancassandra, bigtable}. Seu modelo de dados pode ser visto como um mapa multidimensional indexado por uma chave, se assemelhando aos modelos de chave-valor e orientados à colunas. A seguir veremos em detalhes cada um desses conceitos.

\subsection*{\emph{Keyspace}}
Um \emph{keyspace} define o maior agrupamento de dados no Cassandra, podendo ser correspondido a um banco de um SGBD relacional. Um \emph{keyspace} define um nome e uma série de atributos que definem o seu comportamento~\cite{cassandraguide}.

Atributos do \emph{keyspace} incluem:
\begin{itemize}
\item \textbf{Fator de replicação} diz respeito ao número de nós que armazenarão uma réplica de cada linha de dados. O fator de replicação tem forte influencia no balanço entre performance e consistência do banco de dados.
\item \textbf{Estratégia de replicação} se refere a como as réplicas (ou cópias) de um dado serão posicionados no anel do \emph{cluster}.
\item \textbf{Famílias de colunas} pode ser visto como o análogo às tabelas de um modelo relacional, da mesma forma que o \emph{keyspace} é o análogo do banco.  Uma família de colunas é um agrupamento para uma coleção de linhas, onde cada linha contém colunas ordenadas.
\end{itemize}

\subsection*{Colunas e famílias de colunas}
Uma família de colunas (ou tabela) no Cassandra é um mapa multidimensional indexado por uma chave. Essa chave é uma \emph{string} sem restrição de tamanho, mas que em geral varia de 16 a 36 \emph{bytes}. O valor desse mapeamento consiste em uma família de colunas, um agrupamento para uma coleção ordenadas de linhas, que por sua vez é uma coleção ordenada de colunas ~\cite{lakshmancassandra, cassandraguide}.

O modelo de família de colunas se diferencia do modelo relacional por ser o que é chamado comumente de \emph{livre de esquema} (\emph{schama free}) . É possível realizar a inserção, remoção ou alteração de qualquer coluna ou família de colunas a qualquer momento, ficando as aplicações clientes do banco encarregadas de interpretar e manipular o novo modelo de dados. 

Ao se inserir um novo dado em uma família de colunas do Cassandra são especificados valores para uma ou mais colunas. O conjunto de valores é chamado de linha, e é identificado unicamente por uma chave de linha. Uma linha não precisa possuir dados para todas colunas presentes na família de colunas à que ela pertence, sendo o espaço alocado apenas para as colunas presentes nessa linha. Isso gera tanto uma economia de espaço quanto uma melhora de performance em relação a um banco relacional, que precisa preencher com valores nulos colunas não utilizadas.
