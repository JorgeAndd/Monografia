\chapter{Cassandra}

Esse trabalho ira utilizar o Cassandra como banco de dados para validação de sua hipótese. Como visto no capítulo 3, o Apache Cassandra, distribuição que sera utilizada, é um banco de dados orientado a colunas altamente disponível e distribuído em servidores constituídos de hardware de \enquote{prateleira} para gerenciamento de grande volumes de dados~\cite{lakshmancassandra}. Este capítulo tem como objetivo definir esse banco de dados, suas características, funcionamento, vantagens e desvantagens.

\section{Definição}
O Cassandra se originou em 2007 como um projeto do \emph{Facebook} para resolver um problema na busca da caixa de mensagens. A compania necessitava de um sistema com alta performance, confiabilidade, eficiência e que suportasse o contínuo crescimento da ferramenta~\cite{lakshmancassandra, cassandraguide}. 

O projeto foi desenvolvido por Jeff Hammerbacher, Avinash Lakshman, Karthik Ranganathan e Prashant Malik, tendo seu modelo de dados sofrido grande inspiração nos trabalhos anteriores do \emph{Amazon Dynamo}~\cite{dynamo} e do \emph{Google Bigtable}~\cite{bigtable}, e lançado em 2008 como um projeto \emph{open source}. Foi mantido e atualizado apenas pelo Facebook até 2009, quando foi comprado pela Apache~\cite{cassandraguide}, sendo utilizado atualmente por companias como \emph{Netflix}, \emph{Spotify} e até em agências governamentais, como a NASA~\cite{cassandracompanies}. 

Pode ser definido como um banco de dados orientado a colunas \emph{open source}, distribuído, descentralizado, elasticamente escalavel, altamente disponível, tolerante a falhas e variavelmente consistente~\cite{cassandraguide}. A seguir iremos analisar cada uma dessas características.

\subsection{Características}

\subsection*{Distribuído e Descentralizado}
O Cassandra é capaz de ser executado em múltiplas de forma transparente ao usuário, que o enxerga como um sistema unificado. Apesar de ser possível sua execução em um único nó, só é possível obter algum benefício com uma execução distribuída. Além do ganho de performance, a distributividade do sistema garante maior segurança devido à redundância de dados.

Diferente de outros bancos distribuidos que elegem nós como mestres e escravos, o Cassandra opera de forma descentralizada, o que significa que todos os nós são idênticos em sua forma de execução, sendo utilizados protocolos \emph{peer-to-peer} (par-a-par) e \emph{gossip} para manutenção e sincronia entre os nós. Essa descentralização garante que não exista apenas um ponto de falha, o que aumenta sua disponibilidade, e simplifica a operação do e manutenção do \emph{cluster}.

\subsection*{Elasticamente Escalável}
Escalabilidade é a propriedade que um sistema tem de atender um crescente número de requisições sem prejuízo de performance. Essa escalabilidade pode ser tanto vertical quanto horizontal. Na escalabilidade vertical o hardware já utilizado no sistema é melhorado, enquanto na escalabilidade horizontal novos máquinas são adicionadas à arquitetura, havendo a divisão da carga do sistema.

O Cassandra possui escalabilidade horizontal elástica, o que significa que sua arquitetura pode escalar tanto para cima quanto para baixo. Na necessidade de uma melhora do desempenho da aplicação, novas máquinas podem ser adicionadas, e o Cassandra se encarrega de fazer a distribuição dos dados de forma transparente, sem necessidade de configurações adicionais ou reiniciamento do sistema. Da mesma forma, em caso de necessidade, máquinas podem ser retiradas do \emph{cluster} sem prejuizo ao todo, devido ao rebalanceamento automático.

\subsection*{Altamente disponível e Tolerante a falhas}
A disponibilidade de um sistema é medida de acordo com sua capacidade de responder a requisições. Computadores, e especialmente sistemas distribuídos em rede, estão sujeitos a falhas, que em geral só podem ser contornadas por meio de sistemas redundantes.

Devido a replicação e redundância de dados e a sua capacidade de substituição de nós indisponíveis, o Cassandra pode ser definido como um sistema altamente disponível e tolerante à falhas em suas máquinas.

\subsection*{Variavelmente Consistente}
A consistência de uma aplicação diz respeito à sua capacidade de retornar o valor mais atual em uma requisição.

Como visto no Teorema CAP\ref{sec:cap}, não é possível a um sistema ser totalmente conscistente, disponível e tolerante a falhas. 

O Cassandra é por vezes definido como \enquote{eventualmente consistente}, por trocar parte de sua consistência por alta disponibilidade. Essa definição, porém, não é totalmente correta, e um termo melhor para defini-lo é \enquote{variavelmente consistente} (\emph{tuneably consistent}), podendo essa sua consistência ser ajustada de acordo com o tipo de aplicação.

\section{Modelo Orientado a Colunas}

Uma tabela do Cassandra consiste em um mapa multidimensional indexado por uma chave. Essa chave é uma \emph{string} sem restrição de tamanho, mas que em geral varia de 16 a 36 \emph{bytes}, e o valor do mapa é um objeto altamente estruturado~\cite{lakshmancassandra}.

