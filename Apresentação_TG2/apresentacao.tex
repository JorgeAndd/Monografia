%%%%%%%%%%%%%%%%%%%%%%%%%%%%%%%%%%%%%%%%%
% Beamer Presentation
% LaTeX Template
% Version 1.0 (10/11/12)
%
% This template has been downloaded from:
% http://www.LaTeXTemplates.com
%
% License:
% CC BY-NC-SA 3.0 (http://creativecommons.org/licenses/by-nc-sa/3.0/)
%
%%%%%%%%%%%%%%%%%%%%%%%%%%%%%%%%%%%%%%%%%

%----------------------------------------------------------------------------------------
%	PACKAGES AND THEMES
%----------------------------------------------------------------------------------------


\documentclass[brazil]{beamer}

\usepackage{graphicx} % Allows including images
\usepackage{booktabs} % Allows the use of \toprule, \midrule and \bottomrule in tables
\usepackage[utf8]{inputenc}
\usepackage[T1]{fontenc}
\usepackage[brazil]{babel}
\usepackage{subfig}
\usepackage{xcolor,graphicx,tikz,url}
\usepackage{pbox}
\usepackage{adjustbox}
\usepackage{listings, lstautogobble}

\renewcommand{\lstlistingname}{Código}

\lstset{
	frame=single,
	breaklines=true,
	postbreak=\raisebox{0ex}[0ex][0ex]{\ensuremath{\color{red}\hookrightarrow\space}},
	autogobble=true
}

\mode<presentation> {

% The Beamer class comes with a number of default slide themes
% which change the colors and layouts of slides. Below this is a list
% of all the themes, uncomment each in turn to see what they look like.

\usetheme{default}
%\usetheme{beamertheme-bjeldbak/bjeldbak}
%\usetheme{AnnArbor}
%\usetheme{Antibes}
%\usetheme{Bergen}
%\usetheme{Berkeley}
%\usetheme{Berlin}
%\usetheme{Boadilla}
%\usetheme{CambridgeUS}
%\usetheme{Copenhagen}
%\usetheme{Darmstadt}
%\usetheme{bjeldbak}
%\usetheme{Dresden}
%\usetheme{Frankfurt}
%\usetheme{Goettingen}
%\usetheme{Hannover}
%\usetheme{Ilmenau}
%\usetheme{JuanLesPins}
%\usetheme{Luebeck}
%\usetheme{Madrid}
%\usetheme{Malmoe}
%\usetheme{Marburg}
%\usetheme{Montpellier}
%\usetheme{PaloAlto}
%\usetheme{Pittsburgh}
%\usetheme{Rochester}
%\usetheme{Singapore}
%\usetheme{Szeged}
%\usetheme{Warsaw}

% As well as themes, the Beamer class has a number of color themes
% for any slide theme. Uncomment each of these in turn to see how it
% changes the colors of your current slide theme.

%\usecolortheme{albatross}
%\usecolortheme{beaver}
%\usecolortheme{beetle}
%\usecolortheme{crane}
%\usecolortheme{dolphin}
%\usecolortheme{dove}
%\usecolortheme{fly}
%\usecolortheme{lily}
%\usecolortheme{orchid}
%\usecolortheme{rose}
%\usecolortheme{seagull}
%\usecolortheme{seahorse}
%\usecolortheme{whale}
%\usecolortheme{wolverine}

%\setbeamertemplate{footline} % To remove the footer line in all slides uncomment this line
%\setbeamertemplate{footline}[page number] % To replace the footer line in all slides with a simple slide count uncomment this line

%\setbeamertemplate{navigation symbols}{} % To remove the navigation symbols from the bottom of all slides uncomment this line
}


%----------------------------------------------------------------------------------------
%	TITLE PAGE
%----------------------------------------------------------------------------------------

\title[Bolsa Família e Cassandra]{Armazenamento de Dados Abertos com NoSQL: Um estudo de caso com Dados do Bolsa Família e NoSQL Cassandra} % The short title appears at the bottom of every slide, the full title is only on the title page

\author{Jorge Luiz Andrade} % Your name
\institute[UnB] % Your institution as it will appear on the bottom of every slide, may be shorthand to save space
{
Universidade de Brasília \\ % Your institution for the title page
\medskip
\textit{jorgeluizandrade@outlook.com} % Your email address
}
\date{\today} % Date, can be changed to a custom date

\begin{document}

\begin{frame}
\titlepage % Print the title page as the first slide
\end{frame}

% \begin{frame}
% \frametitle{Sumário} % Table of contents slide, comment this block out to remove it
% \tableofcontents % Throughout your presentation, if you choose to use \section{} and \subsection{} commands, these will automatically be printed on this slide as an overview of your presentation
% \end{frame}

%----------------------------------------------------------------------------------------
%	PRESENTATION SLIDES
%----------------------------------------------------------------------------------------

%------------------------------------------------
\section{Introdução} 
%------------------------------------------------

\begin{frame}
Bancos não relacionais, conhecidos como NoSQL, tem se tornado uma alternativa para o armazenamento de grandes volumes de dados.
	
\begin{block}{Problema}
	Banco de Dados Relacionais podem não apresentar um desempenho satisfatório ao operar grandes volumes de dados.
\end{block}

\begin{block}{Hipótese}
	O uso de múltiplas máquinas em um ambiente Cassandra distribuído pode oferecer um melhora do desempenho que justifique sua utilização na análise de dados abertos.	
\end{block}

\end{frame}

\begin{frame}
\frametitle{Introdução}

\begin{block}{Objetivos}
	Comparar o desempenho de um banco Cassandra para inserções e consultas em diferentes tamanhos de \emph{cluster} e de volumes de dados;
	
	\begin{itemize}
		\item Desenvolver uma aplicação para inserção e busca dos dados do Bolsa Família;
		\item Realizar testes de inserção e busca com diferentes configurações;
		\item Comparar o desempenho do Cassandra nas diferentes situações;
	\end{itemize}
\end{block}
\end{frame}

%------------------------------------------------
\section{Fundamentação Teórica}
%------------------------------------------------

\subsection{Dados Abertos}

\begin{frame}
	\frametitle{Dados Abertos}
	\begin{block}{Contextualização e Características}
		\begin{itemize}
			\item Conceito de dados abertos surgiu em 1995, no contexto de abertura de dados geofísicos e ambientais;
			\item \emph{Open Knowledge Foundation} define um dado como aberto se qualquer pessoa está livre para acessa-lo, utiliza-lo, modifica-lo e compartilha-lo;
		\end{itemize}
	\end{block}
\end{frame}
	
\begin{frame}
	\frametitle{Dados Abertos}

	\begin{block}{Classificação}
		Tim Berners-Lee propõs em 2010 o princípio de cinco estrelas para classificação de dados abertos:
		\begin{itemize}
			\item \textbf{1 estrela}: O dado está disponível na Internet, em qualquer formato, acompanhado de licença aberta.
			\item \textbf{2 estrelas}: O dado está disponível de maneira estruturada, em um formato que permita sua leitura por máquinas (por exemplo, XML).
			\item \textbf{3 estrelas}: Deve estar em formato não proprietário (por exemplo, \emph{CSV}).
		\end{itemize}
	\end{block}
\end{frame}

\begin{frame}
	\frametitle{Dados Abertos}
	
	\begin{itemize}
		\item \textbf{4 estrelas}: Deve estar dentro dos padrões estabelecidos pela W3C (RDF): utilização de URI para nomear relações entre coisas e propriedades.
		\item \textbf{5 estrelas}: Ter no RDF conexões com outros dados, por meio de \emph{links}, de forma a se obter um contexto de dados relevantes.
	\end{itemize}
\end{frame}

\begin{frame}
	\frametitle{Dados Abertos}
	\begin{block}{Contexto Brasileiro}
		\begin{itemize}
			\item \emph{Open Government Partnership}, 2011;
			\item Portal dados.gov.br;
			\item INDA(Infraestrutura Nacional de Dados Abertos), 2012;			
		\end{itemize}
	\end{block}
\end{frame}

%------------------------------------------------

\subsection{Bancos de Dados}

\begin{frame}
\frametitle{Bancos Relacionais}
	\begin{block}{Definição}
		\begin{itemize}
			\item Proposto em 1970 por Edgar Codd;
			\item Composto por um conjunto de relações, sendo essas relações um conjunto não ordernado de tuplas;
			\item SGBDs;
		\end{itemize}
	\end{block}
\end{frame}

\begin{frame}
\frametitle{Propriedades ACID}
	\begin{itemize}
		\item Atomicidade;
		\item Consistência;
		\item Isolamento;
		\item Durabilidade;
	\end{itemize}
	Garantem a validade do esquema, mas sacrificam desempenho e disponibilidade.
\end{frame}


\begin{frame}
\frametitle{Normalização}
	\begin{itemize}
		\item \textbf{1FN}: Cada coluna deve guardar apenas uma informação(valores atômicos);
		\item \textbf{2FN}: Atributos não-chave devem depender integralmente da chave primária da tabela;
		\item \textbf{3FN}: Atributos não-chave não podem ser determinados por outros atributos não-chave; 
	\end{itemize}
\end{frame}

\subsection{NoSQL}

\begin{frame}
\frametitle{NoSQL}
	Modelos relacionais possuem restrições, como as propriedades ACID e Normalização, gerando problemas de escalabilidade e rigidez de esquema.
	\begin{itemize}
		\item Termo utilizado pela primeira vez em 1998(Strozzi NoSQL)
		\item Google Bigtable(2006) e Amazon's Dynamo(2007)
		\item Evitam complexidade desnecessária;
		\item Buscam alto rendimento, escalabilidade e disponibilidade;
		\item \emph{Open Source};
	\end{itemize}	
\end{frame}


\begin{frame}
	\frametitle{NoSQL}
	\begin{block}{Teorema CAP}
	\begin{itemize}
		\item Proposto em 2000 por Eric Brewer, define limitações em sistemas distribuídos;
		\item Consistência;
		\item Disponibilidade;
		\item Tolerância a partições;
		\bigskip
		\item Revisado em 2012;
	\end{itemize}
	\end{block}
\end{frame}

\subsection{Modelos NoSQL}
\begin{frame}
	\frametitle{Modelos NoSQL}
	
	\begin{block}{Chave-Valor}
		Consiste em uma tabela \emph{hash}, com consultas a um valor a partir de uma chave.
		\begin{itemize}
			\item Berkeley DB;
			\item Amazon DynamoDB;
		\end{itemize}
	\end{block}

	\begin{block}{Documentos}
		Acesso à um documento de esquema flexível a partir de uma chave.
	
		\begin{itemize}
			\item CouchDB;
			\item MongoDB;
		\end{itemize}
	\end{block}

\end{frame}


\begin{frame}
	\frametitle{Modelos NoSQL}
	
	\begin{block}{Grafos}
		Dados altamente conectados, com consultas baseadas em relacionamentos.
		\begin{itemize}
			\item Neo4j
			\item OrientDB
		\end{itemize}
	\end{block}

	\begin{block}{Colunas}
		Dados armazenados em famílias de colunas. Possui esquema flexível, permitindo a modificação de colunas a qualquer momento.
		\begin{itemize}
			\item HBase
			\item \textbf{Cassandra}
		\end{itemize}
	\end{block}

\end{frame}


%------------------------------------------------
\subsection{Cassandra}
\begin{frame}
	\frametitle{Cassandra}
	\begin{itemize}
		\item Criado em 2007 pelo \emph{Facebook}, buscando alta performance, confiabilidade, eficiência e que suportasse  contínuo crescimento;
		\item Aberto em 2008 e adotado pela Apache em 2009;
	\end{itemize}
\end{frame}

\begin{frame}
	\frametitle{Cassandra}
	\begin{block}{Características}
		\begin{itemize}
			\item \textbf{Distribuído e Descentralizado}: Execução em múltiplas máquinas, utilizando protocolos \emph{peer-to-peer};
			
			\item \textbf{Elasticamente Escalável}: Suporta adição e remoção de máquinas de forma transparente;
			
			\item \textbf{Altamente disponível e Tolerante a falhas}: Replicação e redundância de dados;
			
			\item \textbf{Variavelmente consistente}: Consistência ajustada por aplicação;
		\end{itemize} 
	\end{block}
\end{frame}

\begin{frame}
	\frametitle{Cassandra}
	\begin{block}{Características}
		\begin{itemize}
			\item \emph{Keyspace} contendo famílias de colunas, que definem um conjunto de linhas englobando várias colunas;
			\item Linguagem CQL, introduzida na versão 0.8;
		\end{itemize}
	\end{block}
\end{frame}
%------------------------------------------------
\section{Metodologia}
\begin{frame}
	\frametitle{Metodologia}
	\begin{block}{Programa Bolsa Família}
		Programa de transferência de renda criado em 2003. 
		
		Em 2016, atendia 13,9 milhões de famílias, que recebiam uma média de R\$ cada, totalizando R\$27,4 bilhões.
		
		\begin{itemize}
			\item Dados disponibilizados no Portal da Transparência;
			\item Arquivos mensais em formato .csv;
		\end{itemize}
	\end{block}
\end{frame}

\begin{frame}
	\frametitle{Metodologia}
	\begin{block}{Dados Utilizados}
		Foram utilizados um total de trinta arquivos, referentes aos meses de Julho de 2014 a Dezembro de 2016. Os arquivos totalizam 16Gib de tamanho e cerca de 14 mil registros.
		
		\begin{table}
			\centering
			\adjustbox{max height=\dimexpr\textheight-7cm\relax,
				max width=\textwidth}{
			\begin{tabular}{|l|c|c|}
				\hline
				\textbf{Campo}         & \textbf{Tipo} & \textbf{Utilizado} \\ \hline
				UF                     & Text          & Sim                \\ \hline
				Código SIAFI Município & Int           & Sim                \\ \hline
				Nome Município         & Text          & Sim                \\ \hline
				Código Função          & -             & Não                \\ \hline
				Código Subfunção       & -             & Não                \\ \hline
				Código Programa        & -             & Não                \\ \hline
				Código Ação            & -             & Não                \\ \hline
				NIS Favorecido         & Bigint        & Sim                \\ \hline
				Nome Favorecido        & Text          & Sim                \\ \hline
				Fonte-Finalidade       & Text          & Sim                \\ \hline
				Valor Parcela          & Double        & Sim                \\ \hline
				Mês Competência        & Timestamp     & Sim                \\ \hline
			\end{tabular}
			}
		\end{table}
	\end{block}
	
\end{frame}

\begin{frame}[fragile]
	\frametitle{Metodologia}
	\begin{block}{Modelo de Dados}
		\begin{itemize}
			\item Fator de replicação de 1 (sem tolerância a falhas);
			\item \emph{SimpleStrategy} (\emph{datacenter} único);
			\item Criação do ambiente com uso de CQL;
		\end{itemize}
	
		\begin{tabular}{c}
		\begin{lstlisting}[caption={Código CQL para criação do keyspace},label={lst:cql_create_table},language=SQL]
		CREATE KEYSPACE bolsa_familia WITH replication = {'class': 'SimpleStrategy', 'replication_factor': 1};
		\end{lstlisting}
		\end{tabular}
	\end{block}
\end{frame}

\begin{frame}[fragile]
	\frametitle{Metodologia}
		\begin{tabular}{c}
		\begin{lstlisting}[caption={Código CQL para criação da tabela},label={lst:cql_create_table},language=SQL]
			CREATE TABLE bolsa_familia.dados (uf TEXT, periodo TIMESTAMP, valor DOUBLE, nis_favorecido BIGINT, cod_municipio INT, fonte TEXT, nome_favorecido TEXT, nome_municipio TEXT, PRIMARY KEY(nis_favorecido, periodo, valor));
		\end{lstlisting}
		\end{tabular}
\end{frame}

\begin{frame}
	\frametitle{Metodologia}

	\begin{block}{Arquitetura do Ambiente}
		\begin{itemize}
			\item \emph{Cluster} composto por seis máquinas Intel i5-4570 3.20GHz, 16GB de RAM, disco rígido de 500GB, com sistema operacional Ubuntu;
			\item Cliente Cassandra versão 3.0.4;
			\item Configuração do arquivo \emph{cassandra.yaml};
		\end{itemize}
	
		Configurações do Linux:
		\begin{itemize}
			\item Remoção do limite de memória;
			\item Aumento do limite do número de arquivos abertos;
			\item Desativação do \emph{swap};
		\end{itemize}
	\end{block}
\end{frame}	

\begin{frame}[fragile]
	\frametitle{Metodologia}
		\begin{tabular}{c}
			\begin{lstlisting}[caption={Configuração cassandra.yaml},language=python]
			cluster_name: 'BolsaFamilia Cluster C2M FR1'
			
			num_tokens: 256
			
			partitioner: org.apache.cassandra.dht.Murmur3Partitioner
			
			seed_provider:
			- class_name: org.apache.cassandra.locator.SimpleSeedProvider
			parameters:
			- seeds: "164.41.40.35"
			
			endpoint_snitch: SimpleSnitch
			\end{lstlisting}
		\end{tabular}
\end{frame}

\begin{frame}
	\frametitle{Metodologia}
	
	\begin{block}{Desenvolvimento da Aplicação}
		Foi desenvolvida uma aplicação em Java responsável pela leitura dos arquivos de entrada, inserção no banco e busca de dados:
		\begin{itemize}
			\item \emph{Driver} \emph{Datastax};
			\item Tratamento e filtragem dos campos;
			\item Interações com o banco por meio de CQL;
		\end{itemize}
	
		 \begin{figure}
			\includegraphics[width=0.7\linewidth]{figuras/aplicacao.png}
			\caption{Esquema da aplicação desenvolvida}
		\end{figure}
	\end{block}
\end{frame}

%------------------------------------------------
\section{Resultados}

\subsection{Carga de Dados}
\begin{frame}
	\frametitle{Resultados}
	\begin{block}{Carga dos Dados}
		A aplicação desenvolvida realiza a filtragem dos campos e tratamento dos valores:
		\begin{itemize}
			\item Remoção do separador de milhares(,) em Valor Parcela;
			\item Alteração do padrão de data de MM/AAAA para DD/MM/AAAA;		
		\end{itemize}
	
		Foi realizada a carga com dois volumes de dados, correspondentes a dezoito e trinta meses do programa Bolsa Família.
		\begin{table}
			\centering
			\caption{Volume de dados}
			\adjustbox{max height=\dimexpr\textheight-7cm\relax,
				max width=\textwidth}{
				\begin{tabular}{ll}
				\textbf{Carga} & \textbf{Tamanho} \\ \hline
				18 meses       &  8,79 GB         \\ \hline
				30 meses       &  14,69 GB        \\ \hline
				\end{tabular}
			}
		\end{table}	
	\end{block}
\end{frame}

\begin{frame}[fragile]
	\frametitle{Resultados}
	\begin{block}{Carga dos Dados}
	Inserção realizada com uso do \emph{driver} da \emph{Datastax}, por meio de query CQL, tendo seus parâmetros substituídos.
	
	\begin{tabular}{c}
		\begin{lstlisting}[caption={Código CQL para inserção},language=SQL]
			INSERT INTO bolsa_familia.dados (uf, cod_municipio, nome_municipio, nis_favorecido, nome_favorecido, fonte, valor, periodo) VALUES (?, ?, ?, ?, ?, ?, ?, ?)
		\end{lstlisting}
	\end{tabular}
	\end{block}
\end{frame}

\begin{frame}
	\frametitle{Resultados}
	\begin{block}{Tempos de Inserção}
	
		\begin{table}
		\centering
		\caption{Tempos de Inserção}
		\adjustbox{max height=\dimexpr\textheight-7cm\relax,
			max width=\textwidth}{
			\begin{tabular}{lllll}
			\textbf{Volume}		& \textbf{2 nós} & \textbf{4 nós} & \textbf{6 nós} \\ \hline
			\textbf{18 meses}   & 1h         	 & 55m       	  & 52m            \\ \hline
			\textbf{30 meses}   & 2h31m      	 & 2h19m          & 2h06m          \\ \hline
			\end{tabular}
		}
		\end{table}	
	
		\begin{table}
			\centering
			\caption{Comparativo}
			\adjustbox{max height=\dimexpr\textheight-7cm\relax,
				max width=0.8\textwidth}{
				\begin{tabular}{lllll}
				\textbf{Volume} 	& \textbf{2 para 4 máquinas} & \textbf{4 para 6 máquinas} & \textbf{Média}  &  \\ \hline
				\textbf{18 meses} 	& 8,70\%                     & 4,22\%                     & \textbf{6,46\%} &  \\ \hline
				\textbf{30 meses}	& 8,13\%                     & 8,94\%                     & \textbf{8,54\%} &  \\ \hline
				\end{tabular}
			}
		\end{table}	
	\end{block}
\end{frame}

\begin{frame}
	\frametitle{Resultados}
	
	\begin{figure}
		\includegraphics[width=0.7\linewidth]{figuras/graphinsert.png}
		\caption{Tempos de Inserção}
	\end{figure}
\end{frame}

\subsection{Consultas}
\begin{frame}[fragile]
	\frametitle{Resultados}
	\begin{block}{Consultas}
	
	As consultas também foram realizadas por meio do \emph{driver} da \emph{Datastax}. Foram realizadas 30 consultas, buscando um registro específico por chave primária de forma aleatória.
	
	\begin{tabular}{c}
		\begin{lstlisting}[caption={Código CQL para consulta},language=SQL]
		SELECT * FROM dados WHERE nis_favorecido = 00020915229557 AND periodo = '2014-07-01' AND valor = 147.00 
		\end{lstlisting}
	\end{tabular}
	\end{block}
\end{frame}

\begin{frame}
	\frametitle{Resultados}
	
	\begin{block}{Tempos de Consulta}
	\begin{table}
		\centering
		\caption{Tempos de Consulta}
		\adjustbox{max height=\dimexpr\textheight-7cm\relax,
			max width=\textwidth}{
			\begin{tabular}{lllll}
				\textbf{Volume}		& \textbf{2 nós} & \textbf{4 nós} & \textbf{6 nós} \\ \hline
				18 meses         & 10,26 s        & 1,95 s        & 1,73 s        \\ \hline
				30 meses         & 12,98 s        & 4,38 s        & 1,43 s         \\ \hline
			\end{tabular}
		}
	\end{table}	
	
	\begin{table}
		\centering
		\caption{Comparativo}
		\adjustbox{max height=\dimexpr\textheight-7cm\relax,
			max width=0.8\textwidth}{
			\begin{tabular}{lllll}
				\textbf{Volume} 	& \textbf{2 para 4 máquinas} & \textbf{4 para 6 máquinas}  & \textbf{Média}   &  \\ \hline
				\textbf{18 meses} 	& 81,00\%                    & 11,41\%                     & \textbf{46,20\%} &  \\ \hline
				\textbf{30 meses}	& 66,21\%                    & 67,48\%                     & \textbf{66,85\%} &  \\ \hline
			\end{tabular}
		}
	\end{table}
	\end{block}	
\end{frame}

\begin{frame}
	\frametitle{Resultados}
	
	\begin{figure}
		\includegraphics[width=0.7\linewidth]{figuras/graphselect.png}
		\caption{Tempos de Consulta}
	\end{figure}
\end{frame}

\begin{frame}
	\frametitle{Resultados}
	
	\begin{block}{Tempos de Consulta}
	O gráfico a seguir apresenta os resultados das dez consultas efetuadas em cada configuração do \emph{cluster} com volume de dados de 18 meses.
	
	\begin{figure}
		\includegraphics[width=0.7\linewidth]{figuras/graphselect_detail.png}
		\caption{Detalhamento dos tempos de consulta}
	\end{figure}
	\end{block}
\end{frame}

%------------------------------------------------
\section{Conclusão}
\subsection{Resultados}
\begin{frame}
\frametitle{Conclusão}
	\begin{block}{Resultados}
		Comparação do aumento do número de máquinas:
		\begin{itemize}
			\item Melhora média de 7,5\% na inserção dos dados;
			\item Melhora média de 56,53\% na busca dos dados;
		\end{itemize}
	\end{block}
\end{frame}

\subsection{Trabalhos futuros}
\begin{frame}
\frametitle{Conclusão}
	\begin{block}{Trabalhos Futuros}
		\begin{itemize}
			\item Isolamento da rede no ambiente utilizado;
			\item Comparação com outros bancos;
			\item Implementar diferentes modelagens no banco Cassandra;
		\end{itemize}
\end{block}
\end{frame}
%------------------------------------------------

%------------------------------------------------
\subsection{Bibliografia}
\begin{frame}
\frametitle{Bibliografia}\footnotesize
  \nocite{seijiconectados, cjdate, pramod, cassandraguide}
  \bibliography{bibliografia}
  \bibliographystyle{plain}
\end{frame}


%----------------------------------------------------------------------------------------

\end{document} 
