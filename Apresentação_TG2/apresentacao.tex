%%%%%%%%%%%%%%%%%%%%%%%%%%%%%%%%%%%%%%%%%
% Beamer Presentation
% LaTeX Template
% Version 1.0 (10/11/12)
%
% This template has been downloaded from:
% http://www.LaTeXTemplates.com
%
% License:
% CC BY-NC-SA 3.0 (http://creativecommons.org/licenses/by-nc-sa/3.0/)
%
%%%%%%%%%%%%%%%%%%%%%%%%%%%%%%%%%%%%%%%%%

%----------------------------------------------------------------------------------------
%	PACKAGES AND THEMES
%----------------------------------------------------------------------------------------


\documentclass[brazil]{beamer}

\usepackage{graphicx} % Allows including images
\usepackage{booktabs} % Allows the use of \toprule, \midrule and \bottomrule in tables
\usepackage[utf8]{inputenc}
\usepackage[T1]{fontenc}
\usepackage[brazil]{babel}
\usepackage{subfig}
\usepackage{xcolor,graphicx,tikz,url}
\usepackage{pbox}
\usepackage[export]{adjustbox}

\mode<presentation> {

% The Beamer class comes with a number of default slide themes
% which change the colors and layouts of slides. Below this is a list
% of all the themes, uncomment each in turn to see what they look like.

\usetheme{default}
%\usetheme{AnnArbor}
%\usetheme{Antibes}
%\usetheme{Bergen}
%\usetheme{Berkeley}
%\usetheme{Berlin}
%\usetheme{Boadilla}
%\usetheme{CambridgeUS}
%\usetheme{Copenhagen}
%\usetheme{Darmstadt}
%\usetheme{bjeldbak}
%\usetheme{Dresden}
%\usetheme{Frankfurt}
%\usetheme{Goettingen}
%\usetheme{Hannover}
%\usetheme{Ilmenau}
%\usetheme{JuanLesPins}
%\usetheme{Luebeck}
%\usetheme{Madrid}
%\usetheme{Malmoe}
%\usetheme{Marburg}
%\usetheme{Montpellier}
%\usetheme{PaloAlto}
%\usetheme{Pittsburgh}
%\usetheme{Rochester}
%\usetheme{Singapore}
%\usetheme{Szeged}
%\usetheme{Warsaw}

% As well as themes, the Beamer class has a number of color themes
% for any slide theme. Uncomment each of these in turn to see how it
% changes the colors of your current slide theme.

%\usecolortheme{albatross}
%\usecolortheme{beaver}
%\usecolortheme{beetle}
%\usecolortheme{crane}
%\usecolortheme{dolphin}
%\usecolortheme{dove}
%\usecolortheme{fly}
%\usecolortheme{lily}
%\usecolortheme{orchid}
%\usecolortheme{rose}
%\usecolortheme{seagull}
%\usecolortheme{seahorse}
%\usecolortheme{whale}
%\usecolortheme{wolverine}

%\setbeamertemplate{footline} % To remove the footer line in all slides uncomment this line
%\setbeamertemplate{footline}[page number] % To replace the footer line in all slides with a simple slide count uncomment this line

%\setbeamertemplate{navigation symbols}{} % To remove the navigation symbols from the bottom of all slides uncomment this line
}


%----------------------------------------------------------------------------------------
%	TITLE PAGE
%----------------------------------------------------------------------------------------

\title[Bolsa Família e Cassandra]{Armazenamento de Dados Abertos com NoSQL: Um estudo de caso com Dados do Bolsa Família e NoSQL Cassandra} % The short title appears at the bottom of every slide, the full title is only on the title page

\author{Jorge Luiz Andrade} % Your name
\institute[UnB] % Your institution as it will appear on the bottom of every slide, may be shorthand to save space
{
Universidade de Brasília \\ % Your institution for the title page
\medskip
\textit{jorgeluizandrade@outlook.com} % Your email address
}
\date{\today} % Date, can be changed to a custom date

\begin{document}

\begin{frame}
\titlepage % Print the title page as the first slide
\end{frame}

% \begin{frame}
% \frametitle{Sumário} % Table of contents slide, comment this block out to remove it
% \tableofcontents % Throughout your presentation, if you choose to use \section{} and \subsection{} commands, these will automatically be printed on this slide as an overview of your presentation
% \end{frame}

%----------------------------------------------------------------------------------------
%	PRESENTATION SLIDES
%----------------------------------------------------------------------------------------

%------------------------------------------------
\section{Introdução} 
%------------------------------------------------
\begin{frame}

\frametitle{Introdução}
\end{frame}

\begin{frame}
\frametitle{Introdução}

\end{frame}

\begin{frame}
\frametitle{Introdução}
\vfill


\begin{block}{Problema}
	Banco de Dados Relacionais podem não apresentar um desempenho satisfatório ao operar grandes volumes de dados
\end{block}
\vfill

\onslide<2->{%
\begin{block}{Hipótese}
	O uso de múltiplas máquinas em um ambiente Cassandra distribuído pode oferecer um melhora do desempenho que justifique sua utilização na análise de dados abertos.	
\end{block}}
\vfill

\end{frame}

\begin{frame}
\frametitle{Introdução}

\begin{block}{Objetivos}
	Comparar o desempenho de um banco Cassandra para inserções e consultas em diferentes tamanhos de \emph{cluster} e de volumes de dados;
	
	\begin{itemize}
		\item Desenvolver uma aplicação para inserção e busca dos dados do Bolsa Família;
		\item Realizar testes de inserção e busca com diferentes configurações;
		\item Comparar o desempenho do Cassandra nas diferentes situações;
	\end{itemize}
\end{block}
\end{frame}

%------------------------------------------------
\section{Fundamentação Teórica}
%------------------------------------------------

% \subsection{Dados Abertos}

% \begin{frame}
% \end{frame}

%------------------------------------------------

\subsection{Bancos de Dados}

\begin{frame}
\frametitle{Bancos Relacionais}
	\begin{itemize}
		\item Proposto em 1970 por Edgar Codd;
		\item Conjunto de relações entre tuplas;
	\end{itemize}
\end{frame}

\begin{frame}
\frametitle{Propriedades ACID}
	\begin{itemize}
		\item Atomicidade;
		\item Consistência;
		\item Isolamento;
		\item Durabilidade;
	\end{itemize}
	Garantem a validade do esquema, mas sacrificam desempenho e disponibilidade.
\end{frame}


\begin{frame}
\frametitle{Normalização}
	\begin{itemize}
		\item \textbf{1FN}: Cada coluna deve guardar apenas uma informação(valores atômicos);
		\item \textbf{2FN}: Atributos não-chave devem depender integralmente da chave primária da tabela;
		\item \textbf{3FN}: Atributos não-chave não podem ser determinados por outros atributos não-chave; 
	\end{itemize}
\end{frame}

\subsection{NoSQL}
Modelos relacionais possuem restrições, como as propriedades ACID e Normalização, gerando problemas de escalabilidade e rigidez de esquema.
\begin{frame}
\frametitle{NoSQL}
	\begin{itemize}
		\item Termo utilizado pela primeira vez em 1998(Strozzi NoSQL)
		\item Google Bigtable(2006) e Amazon's Dynamo(2007)
	\end{itemize}
\end{frame}


\begin{frame}
	\frametitle{Teorema CAP}
	\begin{itemize}
		\item Proposto em 2000 por Eric Brewer, define limitações em sistemas distribuídos;
		\item Revisado em 2012;
		\item Consistência;
		\item Disponibilidade;
		\item Tolerância a partições
	\end{itemize}
\end{frame}

\subsection{Modelos NoSQL}
\begin{frame}
	\frametitle{Chave-Valor}
		Consiste em uma tabela \emph{hash}, com consultas a um valor a partir de uma chave.
		\begin{itemize}
			\item Berkeley DB;
			\item Amazon DynamoDB;
		\end{itemize}
\end{frame}

\begin{frame}
	\frametitle{Documentos}
		Acesso à um documento de esquema flexível a partir de uma chave.
		\begin{itemize}
			\item CouchDB;
			\item MongoDB;
		\end{itemize}
\end{frame}

\begin{frame}
	\frametitle{Grafos}
		Dados altamente conectados, com consultas baseadas em relacionamentos.
		\begin{itemize}
			\item Neo4j
			\item OrientDB
		\end{itemize}
\end{frame}

\begin{frame}
	\frametitle{Colunas}
		Dados armazenados em famílias de colunas. Possui esquema flexível, permitindo a modificação de colunas a qualquer momento.
		\begin{itemize}
			\item HBase
			\item \textbf{Cassandra}
		\end{itemize}
\end{frame}

%------------------------------------------------
\section{Cassandra}


%------------------------------------------------
\section{Resultados}


%------------------------------------------------
\section{Conclusão}
\begin{frame}
\frametitle{Resultados}
	Comparação do aumento do número de máquinas:
\begin{itemize}
	\item Melhora média de 7,5\% na inserção dos dados;
	\item Melhora média de 56,53\% na busca dos dados;
\end{itemize}
\end{frame}

\subsection{Trabalhos futuros}
\begin{frame}
\frametitle{Trabalhos futuros}
\begin{itemize}
	\item Isolamento da rede no ambiente utilizado;
	\item Comparação com outros bancos;
	\item Implementar diferentes modelagens no banco Cassandra;
\end{itemize}


\end{frame}
%------------------------------------------------

%------------------------------------------------
\subsection{Bibliografia}
\begin{frame}
\frametitle{Bibliografia}\footnotesize
  \nocite{}
  \bibliography{bibliografia}
  \bibliographystyle{plain}
\end{frame}


%----------------------------------------------------------------------------------------

\end{document} 
